Le tuilage de données géospatiales connaît une expansion rapide, notamment avec l'essor de la spécification 3D Tiles Next de Cesium. Cette spécification introduit l'Implicit Tiling, permettant la génération automatique de tuiles géospatiales pour le rendu 3D des bâtiments.

L'enjeu est de créer un système efficace pour afficher des bâtiments 3D, en utilisant les données d'OpenStreetMap, de manière optimisée sur le client Cesium.

Le projet vise à développer des ressources compatibles avec Cesium pour une visualisation fluide de bâtiments 3D. Les ressources incluent des fichiers glTF pour les objets 3D et des fichiers Subtree pour la gestion des disponibilités des tuiles.

Les bâtiments sont modélisés en 3D avec différents niveaux de détails grâce au système de Screen Space Error de Cesium. Chaque tuile est indexée via des indices de Morton pour créer des Subtrees, qui sont ensuite hiérarchisés. La structure des Subtrees est paramétrable, permettant d'ajuster les niveaux de détails en fonction des besoins.

Le programme créé permet une génération fonctionnelle et adaptable des Subtrees, optimisant ainsi l'affichage des bâtiments 3D à souhait. Les Subtrees sont transformés en fichiers JSON et buffers binaires pour une transmission efficace.

Pour améliorer ce système, il est recommandé de continuer à affiner la quantité compression des géométries par niveau de détails. Une étude plus approfondie des performances sur différentes plateformes est également suggérée pour trouver le meilleur compromis entre qualité et efficacité. Explorer l'enrichissement des modèles 3D avec des détails supplémentaires comme les toits ou les textures est finalement ce qui manque le plus au programme actuel pour lui donner un aspect plus complet.

% \asterism
