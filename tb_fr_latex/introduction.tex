% L'introduction est une section requise dans un rapport technique. Introduisez votre travail, l'idée de départ et les objectifs attendus. Un lecteur qui découvrirait votre projet au travers de cette introduction devrait ainsi être capable d'en comprendre le cadre, l'idée générale et les aboutissants du projet.

Durant ce travail, j'ai écrit un programme sur la base du \href{https://github.com/baremaps/baremaps}{framework Baremaps}\footnote{https://github.com/baremaps/baremaps} permettant à un client \href{https://cesium.com/}{Cesium}\footnote{https://cesium.com/} d'obtenir les ressources nécessaires à afficher les bâtiments du dataset d'\href{https://www.openstreetmap.org/}{OpenStreetMap}\footnote{https://www.openstreetmap.org/} en 3D.

Ces ressources utilisent la spécification \href{https://cesium.com/blog/2021/11/10/introducing-3d-tiles-next/}{3D Tiles Next}\footnote{https://cesium.com/blog/2021/11/10/introducing-3d-tiles-next/} de Cesium tout en respectant la version 1.1 de leur \href{https://docs.ogc.org/cs/22-025r4/22-025r4.html}{standard de tuiles vectorielles}\footnote{https://docs.ogc.org/cs/22-025r4/22-025r4.html}. Cette nouvelle version de leur spécification introduit entre autre la fonctionnalité de \textit{implicit tiling} \cite{implicit-tiling-gh} qui permet une utilisation facilitée de datasets volumineux. Les ressources qui seront fournies au client Cesium utiliseront donc cette fonctionnalité.

Le programme se divise en deux parties : la première est celle qui s'occupe de la génération des tuiles vectorielles contenant les bâtiments d'OpenStreetMap en 3D et la seconde est celle qui va construire les \textit{Subtrees}, des objects indispensables au client Cesium pour optimiser l'affichage des tuiles. Ces deux parties ne sont pas équivalentes en termes de complexité. La première est plus simple et a été réalisée en premier. La seconde a nécessité d'avantages d'études ainsi que le développement de plusieurs algorithmes pour être implémentée.

Tout le code source de ce projet peut être trouvé sur mon projet GitHub à l'adresse : \href{https://github.com/IEscher/incubator-baremaps-TB-ian}{https://github.com/IEscher/incubator-baremaps-TB-ian}

\newpage
\section*{Fondements écrits par Antoine Drabble}

Afin d'évaluer si le projet s'apprêtais à être utilisé dans le cadre d'un travail de Bachelor, M. Antoine Drabble a effectué un travail de recherche et a produit un prototype de visualisation de bâtiments 3D à l'intérieur du client Cesium JS sur lequel je vais m'appuyer. Ce prototype, bien qu'incomplet, pose les fondements de l'utilisation de Cesium JS, de la spécification 3D Tiles et du framework Baremaps les uns avec les autres. Ce prototype est donc un bon point de départ pour mon travail.

% \section{Contexte}
% Cette section \underline{n'est pas obligatoire}, mais elle est souvent présente dans un rapport technique pour compléter l'introduction et définir le contexte du travail \cad le cadre formel dans lequel le travail est mené.

%%if
% \section{Citations et bibliographie}
% Citer vos sources est essentiel. Avec \texttt{biblatex} vous pouvez facilement citer des articles, des livres ou des sites internet. Toutes les citations dans le texte seront automatiquement regroupées en fin de document dans la section \guillemotleft Bibliographie\guillemotright. Par exemple, citons un article d'Einstein \cite{einstein} ou le livre de Dirac \cite{dirac}.

% Parfois il peut être utile d'utiliser un gestionnaire de bibliographie. La communauté académique recommande l'outil \href{https://www.zotero.org/}{Zotero} qui permet de gérer une bibliothèque numérique d'ouvrages et de références numériques. Il permet également de générer une bibliographie compatible avec \LaTeX.

% Notez qu'il est très facile d'obtenir l'extrait \texttt{bibtex} depuis des journaux. Sélectionnez \emph{export/citation}. Si vous le pouvez choisissez \texttt{bibtex}. Dans le cas d'un format \texttt{.ris}, utilisez un convertisseur en ligne comme \href{http://www.bruot.org/ris2bib/}{ris2bib}.

% \section{Adapter votre modèle}
% Ce document n'est qu'un modèle ayant pour but de revoir les quelques avantages de \LaTeX~ et les fonctionnalités qui pourraient vous être utiles pour rédiger un rapport académique. N'hésitez pas à supprimer les parties inutiles et à adapter ce modèle à vos besoins.
%%fi