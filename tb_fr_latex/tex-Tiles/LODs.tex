Les \textit{Level Of Details}, appellés aussi LOD, est une technique qui permet de définir plusieurs niveaux de détails pour un même objet 3D. Cela permet de charger des objets plus ou moins détaillés en fonction de la distance à laquelle ils se trouvent de la caméra. Grâce à cela, il est possible de réduire la charge de travail de l'ordinateur en ne chargeant que le niveau de détail nécessaire pour garder un rendu qui, à l'oeil humain, semble identique à celui d'un contenant tous les objets entièrement détaillés.

Pour calculer le bon niveau de détail à afficher, Cesium utilise le \texttt{bounding volume} ainsi que le \texttt{geometric error} d'une tuile. Le concept est simple, plus un objet occupe une partie importante de la fenêtre de rendu, plus il doit être détaillé. Le \textit{Screen Space Error} ou SSE est une valeur qui permet de déterminer l'imprécision du rendu d'un objet en Pixels.