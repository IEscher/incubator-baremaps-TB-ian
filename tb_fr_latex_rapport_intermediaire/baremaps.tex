Le framework Baremaps est constitué de plusieurs modules, certains utilisés par tous les projets utilisant le framework, d'autres n'étant utilisés que par certains projets. Parmi les modules principaux, nous avons les modules \texttt{baremaps-cli}, \texttt{baremaps-core}, \texttt{baremaps-server} ainsi que \texttt{basemap}. Parmi les différents points abordés par ce rapport intermédiaire, seuls ces modules seront concernés. Je ne vais donc pas m'attarder ici sur les autres modules du framework.

\section{baremaps-cli}

Le module \texttt{baremaps-cli} est un module permettant de lancer des commandes en ligne de commande pour effectuer des opérations générales avec le framework Baremaps. Ce module est utilisé pour lancer des commandes telles que l'import de données OpenStreetMap ou le lancement d'un serveur web pour afficher des données OpenStreetMap.

\section{baremaps-core}

Le module \texttt{baremaps-core} est un module regroupant la logique métier du framework Baremaps. Ce module est utilisé pour effectuer des opérations telles que la génération de tuiles vectorielles à partir de données OpenStreetMap. C'est dans ce module que la majorité du code qui me servira et que j'écrirai se trouve.

\section{baremaps-server}

Le module \texttt{baremaps-server} est le module qui se charge de lancer un serveur web pour afficher des données OpenStreetMap. Ce module comporte donc les ressources nécessaires pour construire la page web affichant les données OpenStreetMap. C'est dans ce module que Cesium JS sera utilisé et configuré.

\section{basemap}
\label{sec:basemap}

Le module \texttt{basemap} est le module permettant de'importer le fichier OpenStreetMap et de le transformer en une base de données PostgreSQL/PostGIS. À l'heure ou j'écris ce rapport, je n'ai pas encore eu à travailler sur ce module.

\section{Fondements écrits par Antoine Drabble}

Afin d'évaluer si le projet s'apprêtais à être utilisé dans le cadre d'un travail de Bachelor, M. Antoine Drabble a effectué un travail de recherche et a produit un prototype de visualisation de bâtiments 3D à l'intérieur du client Cesium JS sur lequel je vais m'appuyer. Ce prototype, bien qu'incomplet, pose les fondements de l'utilisation de Cesium JS, la spécification 3D Tiles et le framework Baremaps les uns avec les autres. Ce prototype est donc un bon point de départ pour mon travail.

Il n'est pas pertinent de détailler ici le travail effectué par M. Antoine Drabble fichier par fichier mais j'expliquerai au fur et à mesure de ce rapport ce qui a été fait et ce que je dois modifier ou ajouter pour remplir mes objectifs.