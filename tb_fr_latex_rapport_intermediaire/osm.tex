Comme décrit en page \pageref{sec:basemap} \autoref{sec:basemap}, le module Basemap se charge d'importer le fichier OpenStreetMap et de le transformer en une base de données PostgreSQL/PostGIS. Une fois la base de donnée générée, il faut maintenant pouvoir récupérer et traiter les données correctement.

\newpage

\section{Extraction des données et définition d'un bâtiment}
\label{sec:extraction}

La partie nous intéressant dans la base de données OpenStreetMap est la table \texttt{osm\_nodes}. Cette table contient les données de tout les bâtiments. Ceux-ci ne contiennent généralement qu'une \Gls{géométrie}, une structure de donnée contenant plusieurs points et formant un polygone. Parfois seulement, des \Gls{tags} peuvent être ajoutés. Ces tags peuvent contenir des informations sur la hauteur du bâtiment, sa couleur, son matériau, etc.

La \href{https://wiki.openstreetmap.org/wiki/Simple_3D_Buildings}{documentation OpenStreetMap}\footnote{https://wiki.openstreetmap.org/wiki/Simple\_3D\_Buildings} nous permet d'avoir la liste complète des tags existants et nous intéressant concernant les bâtiments.

Pour pouvoir récupérer toutes ces informations, j'utilise la requête SQL suivante :

\begin{verbatim}
SELECT st_asbinary(geom),
    tags -> building,
    tags -> height,
    tags -> building:levels,
    tags -> building:min_level,
    tags -> building:colour,
    tags -> building:material,
    tags -> building:part,
    tags -> roof:shape,
    tags -> roof:levels,
    tags -> roof:height,
    tags -> roof:color,
    tags -> roof:material,
    tags -> roof:angle,
    tags -> roof:direction,
    tags -> amenity
FROM osm_ways
WHERE (tags ? building or tags ? building:part) and
    st_intersects(geom, st_makeenvelope(%1$s, %2$s, %3$s, %4$s, 4326))
LIMIT %5$s;
\end{verbatim}

\texttt{\%1\$s, \%2\$s, \%3\$s, \%4\$s, et \%5\$s} sont des espaces réservés qui seront remplacés par les coordonnées de la zone à charger lors de l'exécution de la requête.

\newpage

\section{Couleurs des bâtiments}

Concernant la couleur des bâtiments, il est possible de récupérer la couleur de base du bâtiment en utilisant le tag \texttt{building:colour}. Cependant, ce tag ne donne que le mot en anglais de la couleur et non pas son code RGB, HEX ou HSL.

Pour remédier à cela, j'ai créé la classe \texttt{ColorUtils}\footnote{baremaps-core/src/main/java/org/apache/baremaps/tdtiles/utils/ColorUtility.java} qui contient une map statique de toutes les couleurs selon le standard définit par \href{https://www.w3schools.com/cssref/css_colors.php}{W3Schools}\footnote{https://www.w3schools.com/cssref/css\_colors.php}. Cette classe permet de convertir le nom de la couleur en code HEX :

\begin{lstlisting}[style=java]
public class ColorUtility {
  private static final Map<String, String> colorMap;

  static {
    colorMap = new HashMap<>();
    colorMap.put("aliceblue", "f0f8ff");
    // ...
    colorMap.put("yellowgreen", "9acd32");
  }

  public static Color parseName(String color) {
    // Calcul des valeurs RGB en fonction du code HEX.
    return new Color(r, g, b);
  }
}
\end{lstlisting}

\texttt{Color}\footnote{baremaps-core/src/main/java/org/apache/baremaps/tdtiles/utils/Color.java} étant un record, il est possible de créer une nouvelle couleur en passant les valeurs RGB en paramètre.

Une fois la couleur RGB récupérée, il est possible de l'utiliser dans la classe \texttt{GltfBuilder}\footnote{baremaps-core/src/main/java/org/apache/baremaps/tdtiles/GltfBuilder.java} pour donner la couleur au bâtiment lors de sa modélisation.

\newpage

\section{Hauteur des bâtiments}

Le calcul de la hauteur des bâtiments doit se faire en fonction de la \href{https://wiki.openstreetmap.org/wiki/Simple_3D_Buildings}{définition}\footnote{https://wiki.openstreetmap.org/wiki/Simple\_3D\_Buildings} des bâtiments par OSM. Si aucun tags ne donne la hauteur du bâtiment, il aura par défaut une hauteur de 10 mètres.

Sinon, la hauteur du bâtiment est calculée en fonction des tags \texttt{height}, \texttt{building:levels}, \texttt{building:min\_level}, \texttt{roof:height} et \texttt{roof:levels} :

\begin{itemize}
    \item On commence avec une hauteur de bâtiment de 0 mètres.
    \item Si aucun tag n'est présent,
          \begin{itemize}
              \item La hauteur du bâtiment est de 10 mètres.
          \end{itemize}
    \item Sinon, si le tag \texttt{height} est présent,
          \begin{itemize}
              \item La hauteur du bâtiment est égale à la valeur du tag \texttt{height}.
              \item Si le tag \texttt{roof:height} est présent,
                    \begin{itemize}
                        \item La hauteur du toit \texttt{roof:height} doit être soustraite de la hauteur du bâtiment
                    \end{itemize}
          \end{itemize}
    \item Sinon,
          \begin{itemize}
              \item Si le tag \texttt{building:levels} est présent,
                    \begin{itemize}
                        \item La hauteur du bâtiment est égale à la valeur du tag \texttt{building:levels} multipliée par 3 mètres.
                    \end{itemize}
              \item Si le tag \texttt{roof:levels} est présent,
                    \begin{itemize}
                        \item La hauteur du toit \texttt{roof:levels} multipliée par 3 mètres doit être additionnée à la hauteur du bâtiment.
                    \end{itemize}
              \item Si le tag \texttt{building:min\_level} est présent,
                    \begin{itemize}
                        \item La hauteur du vide en dessous du bâtiment est égale à la valeur du tag \texttt{building:min\_level} multipliée par 3 mètres.
                    \end{itemize}
          \end{itemize}
\end{itemize}

La hauteur connue, on peut la transmettre à la classe \texttt{GltfBuilder}\footnote{baremaps-core/src/main/java/org/apache/baremaps/tdtiles/GltfBuilder.java} pour donner la hauteur au bâtiment lors de sa modélisation.

\newpage

\section{Géométries des bâtiments}

La classe \texttt{GltfBuilder} a donc accès à la couleur et à la hauteur du bâtiment. Ces informations ainsi que les autres tags lui sont transmis à travers le record \texttt{Building}\footnote{baremaps-core/src/main/java/org/apache/baremaps/tdtiles/building/Building.java} et \texttt{Roof}\footnote{baremaps-core/src/main/java/org/apache/baremaps/tdtiles/building/Roof.java}.

Outre ces informations, \texttt{Building} comporte aussi une \texttt{Geometry}\footnote{org.locationtech.jts.geom.Geometry} qui nous sert à déterminer le polygone au sol du bâtiment. Grâce à ce polygone, nous pouvons extruder le bâtiment en 3D. Cependant une \texttt{Geometry} ne contient qu'une liste de points et il faut une liste de triangles pour pouvoir modéliser le bâtiment en 3D. Pour cela, la méthode \texttt{DelaunayTriangulationBuilder}\footnote{org.locationtech.jts.triangulate.DelaunayTriangulationBuilder} était utilisée. Le problème était que la \href{https://en.wikipedia.org/wiki/Delaunay_triangulation}{\textit{Delaunay triangulation}}\footnote{https://en.wikipedia.org/wiki/Delaunay\_triangulation} ne fonctionne pas avec des polygones concaves. Une version complémentaire de cette méthode est la \href{https://en.wikipedia.org/wiki/Constrained_Delaunay_triangulation}{\textit{Constrained Delaunay triangulation}}\footnote{https://en.wikipedia.org/wiki/Constrained\_Delaunay\_triangulation} qui fonctionne avec des polygones concaves en définissant des segments comme bords du polygone. Cette version de l'algorithme est aussi disponible dans la librairie \href{https://locationtech.github.io/jts/}{JTS}\footnote{https://locationtech.github.io/jts/} mais elle propose aussi la classe \href{https://locationtech.github.io/jts/javadoc/org/locationtech/jts/triangulate/polygon/PolygonTriangulator.html}{\texttt{PolygonTriangulator}}\footnote{https://locationtech.github.io/jts/javadoc/org/locationtech/jts/triangulate/polygon/PolygonTriangulator.html} qui permet aussi de faire une triangulation de polygones concaves de manière moins optimisée mais plus rapide que la \textit{Constrained Delaunay triangulation}. C'est donc cette classe que j'ai utilisée pour trianguler les bâtiments.

Grâce à cette nouvelle triangulation, il est possible de modéliser les bâtiments en 3D correctement, même ceux comprenant des cours intérieures ou des trous dans leur structure.

La classe \texttt{DelaunayTriangulationBuilder} comportait cependant un paramètre \texttt{tolerance} qui permettait de définir la distance minimale entre deux points afin d'optimiser le coût de la triangulation. Cette possibilité n'est pas disponible dans la classe \texttt{PolygonTriangulator}, néanmoins j'implémente une alternative dans le chapitre suivant.